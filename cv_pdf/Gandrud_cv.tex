\documentclass[a4paper]{article}
\usepackage{color}
\usepackage{xcolor}
\usepackage{fullpage}
\usepackage{hyperref}
\hypersetup{
    colorlinks,
    citecolor=black,
    filecolor=black,
    linkcolor=cyan,
    urlcolor=cyan
}
\usepackage{url}
\usepackage[authoryear]{natbib}


\begin{document}

    \begin{flushright}
        {\huge{Christopher Gandrud, Ph.D}} \\
        \vspace{0.251cm}

        \href{mailto:christopher.gandrud@gmail.com}{christopher.gandrud@gmail.com}\\
        \href{https://github.com/christophergandrud}{github.com/christophergandrud}\\[0.25cm]

        ORCID: 0000-0003-4723-7585\\[0.25cm]

        \medskip\hrule height 1pt

        \vspace{0.5cm}

        \today

    \end{flushright}



\vspace{0.5cm}

%%%%%% Positions  %%%%%%%%%%%%%%
\section*{Work Experiences}

\subsection*{Spotify (2022-present)}

\subsubsection*{Senior Manager, Personalization}

Lead a cross-fucntional team of economists, data scientists, and user researchers understanding what drives value for customers on the Spotify personalization experience. \\

\noindent Empathy is at the core of my leadership style. Empathising with customers enables my teams to work on what is important. Empathising with my team members enables me to focus on how to improve our processes to create high impact teams and a great place to work.

\subsection*{Zalando SE (2017-2022)}

\subsubsection*{Director of Economics and Experimentation}

I created and led the Zalando economics teams and supporting engineering and product teams. Over 5 years, the teams were responsible for developing algorithmic ad markets, conducting econometric analyses of strategic company initiatives, and advancing the Zalando A/B testing platform.  

\begin{itemize}

  \item \textbf{Measuring the right things}: I led managers of teams of economists, data scientists, engineers, and product managers redeveloping Zalando's customer centric KPI framework. We use econometric analysis to identify what brings long-term value to diverse customers.

  \item \textbf{Scaling incremental impact research software and services}: Led the development of Zalando's large scale A/B testing platform and broader program to enable teams throughout the company to measure their incremental impact with causal machine learning methods. I guided research into new methods and their application in production.
  
  \item \textbf{Real-time and machine-learned content recommendation system}: We built a platform that dramatically expanded Zalando's on-site advertising business, while improving the customer experience by expanding the diversity and quality of on-site fashion content.

\end{itemize}

\subsection*{Harvard University Institute for Quantitative Social Science (2017)}

\subsubsection*{Research Fellow}

With Prof. Gary King, I led a team developing statistical software for academic and industry applications. I also taught courses on data science and data visualisation.

\subsection*{City, University of London (2015-2018)}

\subsubsection*{Lecturer}

I was a Lecturer in international political economy (quantitative methods) in the Department of International Politics and the Q-Step Centre for quantitative social science education.

\subsection*{Hertie School of Governance (August 2013-June 2017)}

\subsubsection*{Post-Doctoral Fellow}

I coauthored with Prof. Mark Hallerberg a project examining the political economy of responses to financial crises. We were generously funded by the Deutsche Forschungsgemeinschaft (German National Research Foundation).\vspace{0.25cm}

\noindent I designed and taught a graduate-level course on collaborative data science and supervised masters theses.

\subsection*{Yonsei University (2012-2013)}
\subsubsection*{Lecturer}

\subsection*{Hertie School of Governance (2010--2012)}
\subsubsection*{Research Associate}

\subsection*{London School of Economics (2010-2011)}
\subsubsection*{Fellow in Government}

\subsection*{London School of Economics/Peking University (August 2011 \& 2012)}
\subsubsection*{Graduate Teaching Assistant}

\subsection*{London School of Economics (2009-2012)}
\subsubsection*{Graduate Teaching Assistant}

\vspace{0.25cm}
\medskip\hrule height 1pt
\vspace{0.5cm}


%%%%%%% Education %%%%%%%%%%%%%%
\section*{Eduction}

\subsection*{London School of Economics (2008-2012)}
\subsubsection*{MRes/PhD Political Science (Quantitative Research Methodology)}

\emph{Awarded MRes with Distinction} \\

\noindent My thesis used signalling games, event history analysis, and case studies to examine how economic policies are made in times of crisis.

\begin{itemize}
    \item Examiners: Charles Goodhart and Vera E. Troeger
    \item Supervisors: Cheryl Schonhardt-Bailey and Simon Hix
\end{itemize}

\subsection*{London School of Economics (2005-2006)}
\subsubsection*{MSc Comparative Politics (Research Methodology)}

{\emph{Graduated with Distinction}}

\subsection*{McGill University (2003-2005)}
\subsubsection*{BA (Honours) Political Science \& Geography}

{\emph{Graduated with First Class Honours, in the top 10\% of the class}}

\section*{Other Education}

\subsection*{Oxford University Spring Summer School (2014)}

\subsubsection*{Bayesian Hierarchical Models for Social Research (taught by Jeff Gill)}

\vspace{0.25cm}
\medskip\hrule height 1pt
\vspace{0.5cm}

%%%%%%% Publications %%%%%%%%%%%%%%

\section*{Publications}

\subsection*{Published Peer Reviewed Articles}

\begin{itemize}

    \item Taking Time (and Space) Seriously: How Scholars Falsely Infer Policy Diffusion from Model Misspecification. With Cody A. Drolc and Laron K. Williams. 2021 \emph{Policy Studies Journal}. 49(2): 484-515.

    \item The Measurement of Real-Time Perceptions of Financial Stress: Implications for Political Science. With Mark Hallerberg. 2019. \emph{British Journal of Political Science}. 49(4): 1577-1589.

    \item The Reproducibility of Governance Indicators: Assessing Current Practices and Looking Forward. In \emph{Governance Indicators: Approaches, Progress, Promises}. 5th edition. 2018. Oxford: Oxford University Press.

    \item Explaining Variation and Change in Supervisory Confidentiality in the European Union. With Mark Hallerberg. 2018. \emph{West European Politics}. 41(4): 1025-1048.

    \item Financial Regulatory Transparency, International Institutions, and Sovereign Borrowing Costs. With Mark Copelovitch and Mark Hallerberg. 2018. \emph{International Studies Quarterly}. 62(1): 23-41.

    \item Simulating Probabilistic Long-Term Effects in Models with Temporal Dependence. With Laron K. Williams. 2017. \emph{R Journal}. 9(2): 401-408.

    \item Interpreting Fiscal Accounting Rules in the European Union. With Mark Hallerberg. 2017. \emph{Journal of European Public Policy}. 24(6): 832-851.

    \item Information and Financial Crisis Policy Making. With M\'{i}ch\'{e}al O'Keeffe. 2017. \emph{Journal of European Public Policy}. 24(3): 386-405.

    \item Statistical Agencies and Responses to Financial Crises: Eurostat, Bad Banks, and the ESM. With Mark Hallerberg. 2016. {\emph{West European Politics}}. 39(3): 545-564.

    \item Two Sword Lengths Apart: Credible commitment problems and physical violence in democratic national legislatures. 2016. \emph{Journal of Peace Research}. 53(1): 130-145.

        \begin{itemize}
            \item Interviewed about research on \href{http://www.cbc.ca/radio/day6/episode-286-parliamentary-brawls-transgender-rights-the-raptors-secret-weapon-migrant-ships-and-more-1.3591232/you-thought-elbowgate-was-bad-check-out-these-political-brawls-1.3591248}{CBC Radio 1}.

            \item Applied to understanding violence in the Turkish Parliament in \href{https://politicalviolenceataglance.org/2016/05/27/brawl-in-turkish-national-assembly-indicates-deeper-trouble/}{Political Violence @ a Glance}.

            \item Applied to understanding violence in the Ukrainian Parliament in \href{http://voxukraine.org/2015/12/24/causes-and-possible-solutions-to-brawling-in-the-ukrainian-parliament-en-2/}{VoxUkraine}.

            \item Write up on the \href{http://t.co/fETbFCXcYU}{Monkey Cage} with Emily Beaulieu.

        \end{itemize}

    \item Does Banking Union Worsen the EU's Democratic Deficit? The need for greater supervisory data transparency. With Mark Hallerberg. 2015. {\emph{Journal of Common Market Studies}}. 53(4):769-785.

        \begin{itemize}
            \item Data set updated through 2015 at \href{http://bruegel.org/2016/05/the-european-union-remains-a-laggard-on-banking-supervisory-transparency/}{Bruegel}.
        \end{itemize}

    \item When All is Said and Done: Updating `Elections, Special Interests, and Financial Crisis'. With Mark Hallerberg. 2015. \emph{Research and Politics}. 2(3):1-9.

    \item Inflated Expectations: How Government Partisanship Shapes Bureaucrats' Inflation Forecasts. With Cassandra Grafstr\"{o}m. 2015. {\emph{Political Science Research and Methods}}. 3(2):253-380.

    \begin{itemize}
        \item Write up on the \href{http://www.washingtonpost.com/blogs/monkey-cage/wp/2015/01/28/the-fed-cant-accurately-forecast-inflation-heres-why-this-may-hurt-democrats/}{Monkey Cage}.
    \end{itemize}

    \item simPH: An R Package for Illustrating Estimates from Cox Proportional Hazard Models Including for Interactive and Nonlinear Effects. 2015. {\emph{Journal of Statistical Software}}. 65(3):1-20.

    \item Letting German Banks Fail: Federalism and decisions to save troubled banks. With Sahil Deo, Christian Franz, and Mark Hallerberg. 2015 \emph{Politische Vierteljahresschrift} (PVS). 56(2): 159-181.

        \begin{itemize}
            \item Application of the theoretical framework to Austria published on the \href{https://www.weforum.org/agenda/2015/03/a-bank-bailout-lesson-from-austria/}{World Economic Forum}.
        \end{itemize}

    \item Competing Risks and the Mechanisms of Deposit Insurance Governance Convergence. 2014. {\emph{International Political Science Review}}. 35(2):197-217.

    \item The Diffusion of Financial Supervisory Governance Ideas. 2013. {\emph{Review of International Political Economy}}. 20(4):881-916.

\end{itemize}

\subsection*{Peer Reviewed Book}

\begin{itemize}
    \item Reproducible Research with R and RStudio. 2020. Third Edition. {\emph{Chapman \& Hall/CRC Press division of the Taylor \& Francis Group}}.

        \begin{itemize}
            \item First edition reviewed in \emph{The American Statistician}. 2014. 68(4):313--314.
        \end{itemize}

\end{itemize}

\subsection*{Other Published Academic/Policy Articles}

\begin{itemize}

    \item How Not to Create Zombie Banks: lessons for Italy from Japan. with Mark Hallerberg. \emph{Bruegel Policy Contribution}. March 2017.

    \item Visualize Dynamic Simulations of Autoregressive Relationships in R. With Laron K Williams and Guy D Whitten. \emph{The Political Methodologist}. 23(2): 6-10.

    \item Financial Regulatory Transparency: New data and implications for EU policy. With Mark Copelovitch and Mark Hallerberg. \emph{Bruegel Policy Contribution}. December 2015.

    \item Corrections and Refinements to the Database of Political Institutions' \textbf{yrcurnt} Election Timing Variable. 2015. {\emph{The Political Methodologist}}. 22(2): 2-4.

    \item Bad Banks in the EU: The impact of Eurostat rules. With Mark Hallerberg. \emph{Bruegel Working Paper}. 2014/15.

    \item Supervisory Transparency in the European Banking Union. With Mark Hallerberg. {\emph{Bruegel Policy Contribution}}. January 2014.

    \item Who Decides?: Resolving Failed Banks in a European Framework. With Mark Hallerberg. {\emph{Bruegel Policy Contribution}}. November 2013.

      \begin{itemize}
        \item Published in Polish as ``Kto decyduje? Mechanizm Restrukturyzacji i uporzadkowanej likwidacji bank\'{o}w w eropejskich ramach''. 2014. \emph{Nowa Europa}. 1(17): 114-141.
      \end{itemize}

    \item GitHub: A tool for Social Data Set Development and Verification in the Cloud. 2013. {\emph{The Political Methodologist}}. 20(2):2-7.

\end{itemize}


%\section*{Unpublished Working Papers}

%\begin{itemize}

%    \item Predicting Self-fulfilling Financial Crises. With Thomas Pepinsky.

%    \item Bad Banks as a Response to Crises: When Do Governments Use Them, and Why Does Their Governance Differ? With Mark Hallerberg.

%    \item Speaking Under Stress: An Analysis of Federal Reserve Speeches. With Kevin Young.

%    \item Creating Scrutiny Indicators: A Change Point Exploration of Congressional Scrutiny of the US Federal Reserve. With Kevin Young.

%    \item On Whose Account? Financial Crises, Elections, and the Rules of the Game. Book project with Mark Hallerberg.

%    \item Pulling the Trigger: When do governments use macro-prudential policies? With Jeff Chwieroth.

%\end{itemize}

%%%%%%% Teaching %%%%%%%%%%%%%%

\vspace{0.25cm}
\medskip\hrule height 1pt
\vspace{0.5cm}

\section*{Teaching}

\subsection*{RTG DYNAMICS (Humboldt/Hertie)}

\begin{itemize}
    \item Programming Methods for Data Retrieval \& Management
\end{itemize}

\subsection*{Harvard University (Short courses)}

\begin{itemize}
    \item Developing Statistical Software Using IQSS Best Practices
    \item Tools for Reproducible Data Science
    \item Dynamic Web Documents with Dashboards and Shiny (with Dustin Tingley)
\end{itemize}

\subsection*{City University London (Lecturer)}

\begin{itemize}
    \item IP2038: Data Analysis in the Real-World
    \item SG1021: Lies, Damn Lies, and Statistics
    \item SG1022: Producing Social Data
\end{itemize}

\subsection*{Hertie School of Governance (Guest Lecturer)}

\begin{itemize}

    \item MPP-E1180: Introduction to Collaborative Social Science Data Analysis

\end{itemize}

\subsection*{Yonsei University (Lecturer)}

\begin{itemize}

    \item INT3042: Introduction to Social Science Data Analysis
    \item NT4021: The Global Monetary \& Financial System
    \item EIC3120: International Political Economy
    \item INT3041: War, Peacebuilding, and Reconstruction
    \item INT3040: Understanding Global Affairs

\end{itemize}

\subsection*{London School of Economics (Fellow or Teaching Assistant)}

\begin{itemize}
    \item MI452: Applied Regression Analysis
    \item MI451: Introduction to Quantitative Analysis
    \item GV101: Introduction to Comparative Politics
    \item FN208: Post-Crisis: What Next for the International Financial and Monetary Systems? (LSE-Peking University Summer School)
\end{itemize}

%%%%%%% Conference Presentations %%%%%%%%%%%%%%
\vspace{0.25cm}
\medskip\hrule height 1pt
\vspace{0.5cm}

\section*{Conference Presentations/Invited Talks}

\begin{itemize}
    \item Siteman Cancer Centre Biostatistics, Washington University School of Medicine, St. Louis (2022)
    \item PyData Berlin (2018)
    \item American Political Science Association Annual Conference (2012--2017)
    \item European Political Science Association Annual Conference (2011/2014/2016/2017)
    \item LSE Systemic Risk Centre Conference (2016)
    \item School of Politics and International Relations Research Seminar, University of Kent (2016)
    \item International Political Economy Society Conference (2013/2015/2016)
    \item Politics of Secrecy Workshop, University of Munich (2016)
    \item Universit\'{e} de Montr\'{e}al political science talk.
    \item Bank of Iceland: Capital Flows, Systemic Risk, and Policy Responses Conference (2016)
    \item Freie Universit\"{a}t Berlin: Financial Crisis Workshop (2016)
    \item University of Mannheim: Political Forecasting (2015)
    \item Texas A \& M University: Political Budgeting Across Europe (2015)
    \item Open Energy Modeling Workshop (2015)
    \item Political Economy of International Organizations (2015)
    \item Princeton University: Monetary Policy and Central Banking: Historical analysis and contemporary approaches (2015)
    \item Banking Regulation and Supervision and the Great Leap Forward in European Integration, Brussels (2014)
    \item University of Mannheim: Textual Analysis in the Social Sciences (2014)
    \item YoungDNB Seminar on Transparency, Bank of the Netherlands (2014)
    \item csv,Conf (2014)
    \item Midwestern Political Science Association Annual Conference (2012/2014)
    \item CESifo Area Conference on Macro, Money, and International Finance (2014)
    \item International Studies Association Annual Convention (2012/2013)
    \item Waseda University GLOPE II Conference on Political Institutions (2012)
    \item Korean Political Science Association World Congress (2011)
    \item ECPR Regulation in the Age of Crisis (2010)
\end{itemize}


%%%%%%% Other Professional %%%%%%%%%%%%%%
\vspace{0.25cm}
\medskip\hrule height 1pt
\vspace{0.5cm}

\section*{Other Professional Activities}

\section*{Software}

\begin{itemize}
    \item spatialWeights: Calculate spatial weights for time-series cross-sectional data and report spatial clustering test statistics
    \item coreSim: Core functionality for simulating quantities of interest from generalised linear models.
    \item DataCombine: R tools for making it easier to combine and clean data sets.
    \item dpmr: Data package manager for R (with the Open Knowledge Foundation).
    \item dynsim: An R implementation of dynamic simulations of autoregressive relationships (with Laron K Williams and Guy D Whitten).
    \item imfr: R package for interacting with the International Monetary Fund's RESTful JSON API.
    \item networkD3: Tools for creating D3 JavaScript network graphs from R. With JJ Allaire and RStudio.
    \item plotMElm: Plot interaction marginal effects, including with false discovery rate confidence intervals, from linear models.
    \item pltesim: Simulate probabilistic long-term effects in models with temporal dependence (with Laron K Williams).
    \item repmis: A collection of miscellaneous tools for reproducible research with R.
    \item simPH: Tools for simulating and graphing quantities of interest estimated using Cox Proportional Hazards models.
\end{itemize}

\subsection*{Reviewer}

American Political Science Review, Business and Politics, Chapman \& Hall/CRC Press, Economics and Politics,
International Political Science Review, International Studies Perspectives,
International Studies Quarterly,
Journal of Common Market Studies, Journal of Economic Policy Reform, Journal of Statistical Software,
Managerial Finance, Manning Publications Co.,
Packt Publishing, Political Science Research and Methods, Political Research
Quarterly, Regulation and Governance, Review of International Political Economy,
Routledge, Sage Publishing, Scientific Data, West European Politics.

\subsection*{Professional Memberships}

I am or have been a member of the Foundation for Open Access Research, American Political Science Association, the European Political Science Association, and the Midwest Political Science Association, rOpenGov.

\section*{External Funding/Awards}

\begin{itemize}

    \item European Research Council Horizon 2020 (1.5 million Euros, Ronen Palan principal investigator)
    \item LSE IR Department Seed Grant (with Jeff Chwieroth)
    \item Project grant (270,000 Euros) from the Deutsche Forschungsgemeinschaft (with Mark Hallerberg)
    \item LSE Class Teacher Award, Quantitative Methodology (2009-2010)
    \item LSE Government Department PhD Studentship (2008-2012)
    \item Penn State University President's Award
    \item Canadian Tower Scholarship for the Environment

\end{itemize}

%%%%%%%  Other Skills %%%%%%%%%%%%%%
\vspace{0.5cm}
\medskip\hrule height 1pt
\vspace{0.5cm}

\section*{Other Skills}

\subsection*{IT}

I am a highly skilled user of AWS, Bash, CSS, Git/GitHub, HTML, JavaScript, Julia, Jupyter, RMarkdown, LaTeX, Linux, SQL, Python, R, Shiny web apps, SPSS, Stan, and Stata.

\subsection*{Languages}

B1 German proficiency. I have also studied Korean, Spanish, and French.

%%%%%% References %%%%%%%%%%%%%%%%%%%%

\vspace{0.5cm}
\medskip\hrule height 1pt
\vspace{0.5cm}

\section*{References}

\noindent \textbf{Prof. Gary King} \\
1737 Cambridge Street \\
Harvard University \\
Cambridge, MA 02138
USA \\
\href{mailto:king@harvard.edu}{king@harvard.edu}\\

\noindent \textbf{Prof. Mark Hallerberg} \\
Hertie School of Governance\\
Friedrichstra{\ss}e 180\\
Berlin 10117, Germany \\
+49 (0)30 259 219 307 \\
\href{mailto:hallerberg@hertie-school.com}{hallerberg@hertie-school.com}\\

\noindent \textbf{Prof. Jeff Chwieroth}\\
Department of International Relations \\
London School of Economics \\
Houghton Street \\
London WC2A 2AE \\
United Kingdom\\
+44 (0) 20 7955 7209\\
\href{mailto:j.m.chwieroth@lse.ac.uk}{j.m.chwieroth@lse.ac.uk}\\

\noindent \textbf{Prof. Simon Hix}\\
Government Department \\
London School of Economics \\
Houghton Street \\
London WC2A 2AE \\
United Kingdom\\
+44 (0) 20 7955 7657\\
\href{mailto:s.hix@lse.ac.uk}{s.hix@lse.ac.uk}\\

\noindent \textbf{Prof. Cheryl Schonhardt-Bailey}\\
Government Department \\
London School of Economics \\
Houghton Street \\
London WC2A 2AE \\
United Kingdom\\
+44 (0) 20 7955 7187\\
\href{mailto:c.m.schonhardt-bailey@lse.ac.uk}{c.m.schonhardt-bailey@lse.ac.uk}\\

\end{document}
